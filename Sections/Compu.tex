\subsection{Generation of Equilibrium Samples}
Before investigating the non-equilibrium response induced by vibrational excitation, we generated equilibrium reference samples for liquid water, ice Ih, and ice IX using consistent preparation protocols.\\

The liquid water samples were generated from an equilibrium configuration at ambient temperature, which was replicated to obtain a system of one thousand molecules. The system was generated inside a cubic cell, whose side length was chosen to reproduce an initial target mass density of $1 \mathrm{g/cm^3}$.\\

In contrast to the liquid phase, the generation of crystalline ice configurations requires explicit control over proton ordering and lattice symmetry, as well as compliance with the Ice Rules. Under these constraints, crystalline ice samples were generated using the GenIce framework \cite{GenIce}.\\
In order to account for different proton orderings, we chose to represent two ice phases: the Ih phase, which is the stable proton-disordered form of ice under ambient-pressure conditions, and the IX phase, which corresponds to a proton-ordered crystalline structure. For ice Ih, GenIce was used to generate configurations satisfying the ice rules with disordered proton arrangements consistent with the Ih lattice symmetry. For ice IX, a proton-ordered configuration consistent with the corresponding crystallographic structure was generated, without the need for averaging over multiple proton disorder realizations.\\

Due to the constraints imposed by the GenIce framework on the construction of periodic crystalline cells, the ice Ih and ice IX unit cells were replicated to obtain supercells containing 3072 and 2880 atoms, respectively. These system sizes were chosen as the closest multiples compatible with the underlying lattice symmetries and comparable to the system size used for liquid water.


\subsection{Interaction Model and Computational Framework}
All molecular dynamics simulations in this study were performed using a neural network interatomic potential of the NEP (Neuro-Evolution Potential) class. The NEP framework represents the total energy of the system as a sum of atomic contributions expressed as functions of the local environment, with parameters trained on density functional theory (DFT) reference data. This class of potentials has been shown to provide near first-principles accuracy while retaining the computational efficiency required for large-scale molecular dynamics simulations, and has been validated for both liquid water and crystalline ice phases \cite{nep}. The simulations were carried out using the GPUMD package, which provides a native implementation of NEP potentials and enables efficient GPU-accelerated molecular dynamics. All equilibrium and non-equilibrium trajectories discussed in this work were generated within this computational framework.


\subsection{Simulation Protocol}
For each system, an equilibration stage was first performed to ensure thermodynamic stability prior to the excitation procedure. Equilibration was carried out in the isothermal–isobaric (NPT) ensemble. The temperature was controlled using a Nosé–Hoover chain thermostat, while pressure was regulated through a Berendsen barostat, allowing the simulation cell to relax to the target density. Liquid water was equilibrated at $300\,\mathrm{K}$, whereas the crystalline ice phases were equilibrated at $200\,\mathrm{K}$. Equilibrium was assessed by monitoring the stability of the total energy, temperature, and pressure over time.\\

At the end of the equilibration stage, the final configuration was used as the reference structure for the excitation protocol. Quantum nuclear fluctuations were incorporated through a Wigner phase-space representation applied to selected vibrational degrees of freedom.\\
The corresponding distribution was constructed from the eigenstates of the full Lippincott–Schroeder (LS) potential for hydrogen-bonded pairs \cite{lspot}. The one-dimensional Schrödinger equation associated with the LS potential was solved to obtain the vibrational eigenfunctions and eigenvalues, which were used to evaluate the phase-space distribution.\\
A fraction corresponding to 10\% of the total water molecules was randomly selected and initialized quantum mechanically. For each selected molecule, initial positions and momenta were sampled from the phase-space distributions associated with the $n=0$ and $n=1$ vibrational eigenstates of the LS potential for the two O–H bonds, respectively. The remaining 90\% of the molecules were initialized according to the classical Maxwell–Boltzmann distribution at the equilibration temperature.\\

After the excitation step, the system was propagated in the microcanonical (NVE) ensemble to prevent artificial energy exchange with an external thermostat and to enable an unbiased characterization of the intrinsic non-equilibrium response. The subsequent dynamics were followed over a time interval of $30\,\mathrm{ps}$, sufficient to capture both the prompt structural response and the subsequent energy redistribution processes.\\
All trajectories were integrated using a time step of $0.1\,\mathrm{fs}$, chosen to accurately resolve the high-frequency O–H stretching vibrations and to ensure stable energy conservation during the microcanonical production runs.\\

For each phase, the entire equilibration–excitation–production workflow was repeated 50 times independently. All reported structural and dynamical observables were obtained by averaging over these independent realizations, and the associated uncertainties were estimated from the corresponding ensemble fluctuations.