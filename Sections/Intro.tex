Water is a paradigmatic hydrogen-bonded system whose macroscopic properties emerge from a highly cooperative and continuously fluctuating network of intermolecular interactions. Unlike simple liquids, its local structure is neither rigid nor fully random: directional hydrogen bonds impose transient constraints that couple intermolecular organization to intramolecular vibrations. As a result, structural correlations and molecular motions are deeply intertwined, and even localized perturbations can induce collective responses across the network. This interplay persists across phases, from the disordered liquid to crystalline ices with distinct topologies and degrees of proton order, making water an ideal framework to investigate how microscopic excitations propagate in a correlated molecular environment.\\

Experimentally, ultrafast vibrational spectroscopy has provided detailed insight into the nonequilibrium dynamics of water. Infrared pump–probe measurements have revealed sub-picosecond relaxation of the O–H stretching mode and rapid redistribution of excess energy toward bending and librational modes within the hydrogen-bond network \cite{dettori2019energy}. Complementary two-dimensional infrared (2D-IR) spectroscopy has further established a direct connection between vibrational frequency and local hydrogen-bond configuration, enabling the time-resolved observation of spectral diffusion and hydrogen-bond rearrangements \cite{lspot}. In particular, the evolution of 2D line shapes demonstrates that the O–H stretching frequency reflects instantaneous hydrogen-bond geometry and that structural fluctuations occur on femtosecond to picosecond timescales. While these techniques provide exquisite temporal resolution and indirect structural sensitivity, disentangling vibrational energy redistribution from the underlying microscopic reorganization of the hydrogen-bond network remains challenging.\\


The paper is organized as follows: first a brief and general
introduction on the GLE is given, focusing on the frequency-
dependent coupling between the thermostat and the system;
then a first case study, namely, liquid methanol, is presented
together with a detailed systematic analysis of technical aspects
of the thermostat; finally, we report a second application on
liquid water in order to highlight the portability of the method
and the physical differences between the two systems.
